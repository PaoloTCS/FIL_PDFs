% stub
% Chapter 5 – Physical & Computational Correspondence
% ---------------------------------------------------
% Source PDFs:
%   * Particle_Physics_and_Information.pdf (PPI)
%   * Computational Relativity Theory – Complete Discussion and Experimental Design1.pdf
%   * Computational Relativity_ A Framework for Information Processing in Spacetime copy.pdf
%   * Computational Spacetime Geometry.pdf
%   * Chapter 3 Physical Foundations of Computational Light‑Speed.pdf
%   * interferoshell_main.pdf
%   * Path-Encoding-Draft2.pdf
% ---------------------------------------------------

\chapter{Physical and Computational Correspondence}\label{chap:physical}

\section{Particles as Information Events}\label{sec:ppi}

Each fundamental interaction instantiates new information \cite{PPI}.  Let $S(t)=\{p_i,E_{ij}\}$ denote the quantum state at time~$t$.  A collision
$T:S(t)\to S(t+1)$ increases the descriptive complexity
\begin{equation}
  \Delta I \;=\; I\bigl(S(t+1)\bigr)-I\bigl(S(t)\bigr)
  \;\le\; k E.
\end{equation}
This yields a conservation‑of‑information principle analogous to energy conservation in QFT.

\section{Computational Relativity Analogy}\label{sec:comp-rel}

\subsection{Complexity ↔ Time‑Dilation}
Landauer–Bremermann bounds suggest that executing $N$ logical operations on mass~$m$ requires a minimum proper time $\Delta\tau\ge N\hbar/ (mc^2)$.
Interpreting complexity as curvature we obtain a metric on algorithmic spacetime:
\begin{equation}
  ds^2 = -\Bigl(1-\frac{2G\,C}{c^2 r}\Bigr)c^2 dt^2 + \Bigl(1-\frac{2G\,C}{c^2 r}\Bigr)^{-1}dr^2 + r^2 d\Omega^2,
\end{equation}
with $C$ the local Kolmogorov complexity density.

\subsection{Physical Foundations of Computational Light‑Speed}
Chapter~3 of \cite{PhysLightSpeed} derives a physical upper speed of algorithmic propagation that coincides with the semantic constant $\cs$ introduced in Chapter~\ref{chap:geometry}.

\section{Computational Spacetime Geometry}\label{sec:csg}
Information flow defines a causal set $\mathcal C$ whose Hasse diagram embeds into the knowledge graph $G$.  Curvature of $\mathcal C$ matches informational curvature $\kappa$ of Section~\ref{sec:curvature}, closing the semantic‑physical loop.

\section{InterferoShell Hardware Sketch}\label{sec:ishell}

\begin{description}
  \item[Architecture.]  A ring interferometer couples two superconducting qubits via a variable‑delay photonic path.
  \item[Function.]     Physical phase shifts implement the mask evolution operator $T^n_1$ in hardware, stabilising drift.
  \item[Throughput.]   Operational bandwidth is limited by the physical light‑speed bound $c$ and translates to $\cs$ cycles in semantic time.
\end{description}

\section{Prime Path Encoding for Graph Compression}\label{sec:path-encoding}

Path‑encoding maps concept trajectories to unique prime products $\pi(P)=\prod p_i^{n_i}$.  The length satisfies
\begin{equation}
  |P|\;\le\;O\bigl(\log \pi(P)\bigr),
\end{equation}
providing asymptotically optimal storage (\cite{PathEncoding}).

\section{Discussion and Future Work}\label{sec:future}
Links between semantic and physical bounds suggest experimental tests:
\begin{enumerate}
  \item Measure drift suppression in InterferoShell prototypes.
  \item Investigate Hawking‑like emission at semantic black‑hole horizons where $\rho\to\infty$.
  \item Extend computational relativity metrics to non‑commutative graph geometries.
\end{enumerate}

\bigskip\noindent\textbf{Take‑away.}\;Physical interaction, algorithmic complexity, and semantic structure obey the same limiting principles, unifying FIL with fundamental physics.

\clearpage

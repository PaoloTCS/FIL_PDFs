% Chapter 1 – Foundations & Semantic Constants
% (Cleaned version — no placeholder glyphs)

\chapter{Foundations and Semantic Constants}\label{chap:foundations}

\section{The Information--Observation--Language (I--O--L) Triad}\label{sec:IOL-triad}

Modern knowledge systems reveal a recurring motif: comprehension requires only the
\emph{differences} between two bodies of knowledge, not their entirety.  We formalise this
with the \textbf{Information--Observation--Language triad}
\begin{equation}
  \Triad = (\mathcal I,\, \mathcal O,\, \mathcal L),
\end{equation}
where $\mathcal I$ denotes possible information states, $\mathcal O$ the set of admissible
observations, and $\mathcal L$ the symbolic language capable of encoding both.  Minimal
"bridges" between domains are implemented by Local Language Constructors (LLCs), treated
in Chapter~\ref{chap:dynamics}.

\section{Foundational Postulates}\label{sec:axioms}

\begin{description}
  \item[Postulate F1 (Semantic locality).]  Any act of communication factors through a
  finite sub‑language $B\subseteq \mathcal L$ such that $E\otimes B\cong \mathcal L$, with
  $E$ the receiver's existing language fragment.

  \item[Postulate F2 (Minimal bridges).]  Among all such $B$, natural communication
  selects one that minimises $|B|$.

  \item[Postulate F3 (Hierarchical union).]  Languages compose by hierarchical union and
  the semantic density $\rho$ is non‑decreasing under this union.
\end{description}

\section{Semantic Constants}\label{sec:constants}

We introduce two universal constants:
\begin{description}
  \item[$\cs$] the \emph{semantic light‑speed}, bounding information propagation in a
  knowledge graph:
  \begin{equation}
    d_G(v_1,v_2)\;\le\;\cs\,\Delta t.
  \end{equation}

  \item[$\hs$] the \emph{semantic Planck constant}, appearing in an uncertainty relation
  between discovery and invention operators:
  \begin{equation}
    \Delta D\,\Delta I\;\ge\;\tfrac12\hs.\label{eq:sem-uncertainty}
  \end{equation}
\end{description}
Convenient units set $\cs\!=\!\hs\!=\!1$; deviations measure complexity.

\section{Road‑map}
This chapter establishes notation for the remainder of the book.
Chapter~\ref{chap:geometry} develops the geometric view ($\cs$ as cone slope),
Chapter~\ref{chap:bounds} derives global limits from Eq.~\eqref{eq:sem-uncertainty},
Chapter~\ref{chap:dynamics} treats drift and masks, and
Chapter~\ref{chap:physical} links the constants to physical information bounds.

\bigskip\noindent\textbf{Take‑away.}\;The triad $\Triad$ and constants $(\cs,\hs)$ provide
an irreducible substrate on which all higher FIL structures are built.

\clearpage

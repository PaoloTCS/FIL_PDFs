% stub
% Chapter 3 – Informational Bounds & Entropy Limits
% Source PDFs: Bekenstein_Bound_FIL.pdf,
%              Finite_Knowledge_Bounds.pdf,
%              Accelleration.pdf,
%              Big Bang as Information Phase Transition.pdf
% ------------------------------------------------------------

\chapter{Informational Bounds}\label{chap:bounds}

\section{Bekenstein‑like Entropy Limit for Language}

The classical Bekenstein bound \cite{bekenstein1981} constrains the maximum entropy $S$ of physical matter in a region of radius $R$ and energy $E$ by $S\le 2\pi k E R/\hbar c$.  In the FIL setting, tokens carry informational mass and the analogue becomes
\begin{equation}
  H(\mathcal L) \;\le\; 2\pi\,k_s\,E_{\!\mathcal L}\,R_{\!\mathcal L}\,\big/\,\hs,
\end{equation}
where $E_{\!\mathcal L}$ is the cumulative energetic cost of storing the language fragment and $R_{\!\mathcal L}$ its semantic diameter.

\paragraph{Interpretation.}  If $H$ exceeds this limit, further compression (via LLC bridges) or hierarchical segmentation is required.

\section{Finite Knowledge Bounds}

Let $G=(V,E)$ be a directed knowledge graph with path entropy $H_P$ and symbol complexity $C_s$.  \textcite{finiteknowledgebounds2025} derive upper and lower compression bounds
\begin{equation}
  \frac{|V|}{\log C_s}\;\le\;H_P\;\le\;|E|\,\log C_s.
\end{equation}
We adopt the lower bound as the \emph{finite knowledge bound} (FKB) for any segment.

\subsection{Prime‑Encoding Lower Bound}
Prime‑encoding of edge labels achieves the lower bound asymptotically; see Appendix~\ref{app:prime-encoding}.

\subsection{Voronoi Capacity Upper Bound}
Semantic Voronoi cells give a geometric ceiling; we revisit this in Chapter~\ref{chap:dynamics} when studying drift fields.

\section{Acceleration Constraint}

\textcite{accelleration2025} show that rapid semantic updates imply an \emph{acceleration cost}
\begin{equation}
  a_{\!\mathcal L}\;\equiv\;\frac{d^2 H}{dt^2}\;\le\;\cs^2/\ell_{\min},
\end{equation}
with $\ell_{\min}$ the minimum edge length.  This links light‑cone slope ($\cs$) to second‑order dynamics.

\section{Big Bang as Information Phase Transition}
The cosmological Big Bang is recast as a phase transition where $H\to 0$ while $dH/dt\to\infty$.  Under FKB this corresponds to the creation of the minimal language fragment required for any subsequent evolution.

\section{Road‑map}
Informational limits set the stage for Chapter~\ref{chap:dynamics}, where drift and masks operate within these bounds, and Chapter~\ref{chap:physical}, which ties them to particle‑level information exchange.

\clearpage

% stub
% Chapter 2 – Semantic Geometry & Light‑Cone Structure
% Source PDFs: 03_Semantic_Lightcones_and_Propagation.pdf,
%              04_Informational_Curvature_and_Semantic_Gravitation.pdf,
%              Computational Spacetime Geometry.pdf,
%              FL_Field_MetaLaw_to_Quantization.pdf
% ------------------------------------------------------------

\chapter{Semantic Geometry}\label{chap:geometry}

\section{Semantic Light‑Cones}\label{sec:lightcones}

In a knowledge graph $G=(V,E)$ we define the \emph{semantic distance}
$d_G(v_1,v_2)$ as the minimal edge‑weighted path length between two
concept vertices.  Postulate F1 implies that information flow is bounded
by the constant $\cs$ introduced in Chapter~\ref{chap:foundations}:
\begin{equation}
  d_G(v_1,v_2) \;\le\; \cs\,\Delta t.
  \label{eq:semantic-cone}
\end{equation}
Points satisfying equality trace \textbf{semantic light‑cones}.  They
encode the frontier beyond which two agents cannot reach mutual
comprehension within $\Delta t$.

% TODO: Figure — cone diagram in $t$–$d_G$ plane.

\paragraph{Propagation kernel.}  Let $K_t(v)$ denote the reachable set
from $v$ in time $t$.  A discrete propagator is
$P_t=\mathbb 1_{d_G\le\cs t}$.
% (Further derivations to come; see CSLP p. 3.)

\section{Informational Curvature}\label{sec:curvature}

Light‑cone slope alone does not capture \emph{semantic gravitation}—the
tendency of dense subgraphs to attract interpretive trajectories.  We
introduce an \emph{informational curvature tensor} $\mathcal K$ via the
deviation of geodesics in $G$:
\begin{equation}
  \delta^2 d_G = -\mathcal K(\gamma,\dot\gamma)\,d\sigma^2.
\end{equation}
Positive curvature corresponds to semantic "mass" and appears near high
mutual‑information clusters.  See \cite{curvaturePDF}(§2) for the full
derivation.

\subsection*{Example – category junction}
Two dense languages $L_1,L_2$ joined by a minimal LLC bridge $B$ create
negative curvature in the bridge (saddle) and positive curvature inside
each language core.

\section{Computational Spacetime Correspondence}\label{sec:comp-spacetime}

Mapping Eq.~\eqref{eq:semantic-cone} onto a Turing tape with physical
delays $\tau$, we recover the computational light‑speed bound
$\cs\approx\!1/\tau$ \cite{compSpacetime}.  Informational curvature then
corresponds to non‑uniform memory access latencies—regions of high
semantic mass behave as RAM “black holes”.

\section{Meta‑Law and Quantisation}\label{sec:metalaw}

FL\_Field Meta‑Law postulates a universal action principle in
information space.  Quantising small oscillations of $d_G$ about a
ground state yields a discrete spectrum analogous to normal modes in
Riemannian geometry.  This motivates the uncertainty relation derived in
Eq.~(\ref{eq:sem-uncertainty}).

% ------------------------------------------------------------
\bigskip\noindent\textbf{Take‑away.}\;The geometry of semantic light‑cones and
informational curvature generalises relativistic causality to knowledge
systems, setting the stage for global bounds (Chapter~\ref{chap:bounds})
and dynamical drift analysis (Chapter~\ref{chap:dynamics}). 

% ---- 2.5 Voronoi Tessellation for Knowledge Navigation ----
% §2.5 – Voronoi Tessellation for Knowledge Navigation
% ----------------------------------------------------
% Source Chat: "Tessellation for Knowledge Navigation" walk‑notes (o3)

\newtheorem{definition}{Definition}[section]
\chapter{Geometry of Knowledge Systems}\label{chap:geometry}
\section{Voronoi Tessellation and Knowledge Navigation}\label{sec:tessellation}

Traditional path‑finding on a dense knowledge graph $G=(V,E)$ scales poorly with $|V|$.
To achieve sub‑logarithmic query time we partition $V$ into \emph{semantic Voronoi cells}:
\begin{definition}[Semantic Voronoi Cell]
For a seed node $s\in V$ the cell $\operatorname{Vor}(s)$ is
\[
  \operatorname{Vor}(s)=\{v\in V\mid d_G(v,s)\le d_G(v,s')\;\forall s'\neq s\}.\,
\]
\end{definition}
Cells tile the graph when seeds form a maximal $\epsilon$‑net.  The navigation algorithm is then:
\begin{enumerate}
  \item Locate the source and target cells via hashing of node signatures.
  \item Traverse the cell adjacency graph (typically \(O(\sqrt{|V|})\) cells).
  \item Within the target cell, apply local Djikstra (size $\le\epsilon$).
\end{enumerate}

\paragraph{Curvature connection.}  Boundaries between cells coincide with geodesics of
informational curvature $\kappa$ (Section~\ref{sec:curvature}).  High $\kappa$ regions
produce finer tessellations, automatically allocating more seeds where semantic density is high.

\paragraph{Drift boundaries.}  In Chapter~\ref{chap:dynamics} drift masks align to cell
faces; thus tessellation acts as a coarse pre‑mask, reducing the dimensionality of drift
correction.

\paragraph{Complexity.}  With uniform $k$‑nearest‑neighbour degree and $n$ seeds the
cell adjacency graph has $O(n)$ edges; navigation complexity becomes
\(O(\sqrt{n}+\epsilon)\), sub‑logarithmic in $|V|$ for well‑chosen $n\approx\sqrt{|V|}\).

\paragraph{Future work.}  Investigate hyperbolic Voronoi in negative‑curvature regions and
probabilistic tessellations where seed membership is fuzzy.

% end §2.5

% ---------------- 2.5 Voronoi Tessellation ----------------
% §2.5 – Voronoi Tessellation for Knowledge Navigation
% ----------------------------------------------------
% Source Chat: "Tessellation for Knowledge Navigation" walk‑notes (o3)

\newtheorem{definition}{Definition}[section]
\chapter{Geometry of Knowledge Systems}\label{chap:geometry}
\section{Voronoi Tessellation and Knowledge Navigation}\label{sec:tessellation}

Traditional path‑finding on a dense knowledge graph $G=(V,E)$ scales poorly with $|V|$.
To achieve sub‑logarithmic query time we partition $V$ into \emph{semantic Voronoi cells}:
\begin{definition}[Semantic Voronoi Cell]
For a seed node $s\in V$ the cell $\operatorname{Vor}(s)$ is
\[
  \operatorname{Vor}(s)=\{v\in V\mid d_G(v,s)\le d_G(v,s')\;\forall s'\neq s\}.\,
\]
\end{definition}
Cells tile the graph when seeds form a maximal $\epsilon$‑net.  The navigation algorithm is then:
\begin{enumerate}
  \item Locate the source and target cells via hashing of node signatures.
  \item Traverse the cell adjacency graph (typically \(O(\sqrt{|V|})\) cells).
  \item Within the target cell, apply local Djikstra (size $\le\epsilon$).
\end{enumerate}

\paragraph{Curvature connection.}  Boundaries between cells coincide with geodesics of
informational curvature $\kappa$ (Section~\ref{sec:curvature}).  High $\kappa$ regions
produce finer tessellations, automatically allocating more seeds where semantic density is high.

\paragraph{Drift boundaries.}  In Chapter~\ref{chap:dynamics} drift masks align to cell
faces; thus tessellation acts as a coarse pre‑mask, reducing the dimensionality of drift
correction.

\paragraph{Complexity.}  With uniform $k$‑nearest‑neighbour degree and $n$ seeds the
cell adjacency graph has $O(n)$ edges; navigation complexity becomes
\(O(\sqrt{n}+\epsilon)\), sub‑logarithmic in $|V|$ for well‑chosen $n\approx\sqrt{|V|}\).

\paragraph{Future work.}  Investigate hyperbolic Voronoi in negative‑curvature regions and
probabilistic tessellations where seed membership is fuzzy.

% end §2.5

% ---------------- 2.5 Voronoi Tessellation ----------------
% §2.5 – Voronoi Tessellation for Knowledge Navigation
% ----------------------------------------------------
% Source Chat: "Tessellation for Knowledge Navigation" walk‑notes (o3)

\newtheorem{definition}{Definition}[section]
\chapter{Geometry of Knowledge Systems}\label{chap:geometry}
\section{Voronoi Tessellation and Knowledge Navigation}\label{sec:tessellation}

Traditional path‑finding on a dense knowledge graph $G=(V,E)$ scales poorly with $|V|$.
To achieve sub‑logarithmic query time we partition $V$ into \emph{semantic Voronoi cells}:
\begin{definition}[Semantic Voronoi Cell]
For a seed node $s\in V$ the cell $\operatorname{Vor}(s)$ is
\[
  \operatorname{Vor}(s)=\{v\in V\mid d_G(v,s)\le d_G(v,s')\;\forall s'\neq s\}.\,
\]
\end{definition}
Cells tile the graph when seeds form a maximal $\epsilon$‑net.  The navigation algorithm is then:
\begin{enumerate}
  \item Locate the source and target cells via hashing of node signatures.
  \item Traverse the cell adjacency graph (typically \(O(\sqrt{|V|})\) cells).
  \item Within the target cell, apply local Djikstra (size $\le\epsilon$).
\end{enumerate}

\paragraph{Curvature connection.}  Boundaries between cells coincide with geodesics of
informational curvature $\kappa$ (Section~\ref{sec:curvature}).  High $\kappa$ regions
produce finer tessellations, automatically allocating more seeds where semantic density is high.

\paragraph{Drift boundaries.}  In Chapter~\ref{chap:dynamics} drift masks align to cell
faces; thus tessellation acts as a coarse pre‑mask, reducing the dimensionality of drift
correction.

\paragraph{Complexity.}  With uniform $k$‑nearest‑neighbour degree and $n$ seeds the
cell adjacency graph has $O(n)$ edges; navigation complexity becomes
\(O(\sqrt{n}+\epsilon)\), sub‑logarithmic in $|V|$ for well‑chosen $n\approx\sqrt{|V|}\).

\paragraph{Future work.}  Investigate hyperbolic Voronoi in negative‑curvature regions and
probabilistic tessellations where seed membership is fuzzy.

% end §2.5

% ---------------- 2.5 Voronoi Tessellation ----------------
\input{chapters/chap2_geometry/sec25_tessellation.tex}
% ----------------------------------------------------------


% ----------------------------------------------------------


% ----------------------------------------------------------


% -----------------------------------------------------------


\clearpage

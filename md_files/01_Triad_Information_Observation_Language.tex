\input{preamble}
\begin{document}

\section*{The Information–Observation–Language Triad}

This section introduces the foundational conceptual triad of the Semantic Physics framework: \emph{Information}, \emph{Observation}, and \emph{Language}. These three layers interact to produce meaningful structure from latent potential. They mirror the physical world's transition from undistinguished quantum states to classical observables and communicable propositions.

\subsection*{1. Information as Substrate}

We define \textbf{information} as the potential content of a system---its structure, state space, or entropy---prior to measurement or symbolic encoding. In the FIL (Fundamental Interaction Language) framework, this corresponds to the space of all valid (but possibly uninstantiated) relations and transformations between entities.

Mathematically, we model the information substrate as a manifold $\mathcal{I}$ embedded in a semantic Hilbert space:
\[
\mathcal{I} = \{ \psi \in \mathcal{H} : \psi \text{ represents latent semantic configuration} \}
\]

This space is not yet discretized; it contains the possible, the inferable, the conjectural. It is where causality, logic, and potential identity reside, but have not yet been collapsed into experience.

\subsection*{2. Observation as Instantiation}

\textbf{Observation} is the act of resolving a portion of the information substrate into a knowable form. It imposes a local constraint or measurement on $\mathcal{I}$, resulting in a specific state. We define the observation operator $\mathcal{O}$ as:
\[
\mathcal{O} : \mathcal{I} \to \mathcal{K}, \quad \text{where } \mathcal{K} \subset \mathcal{H} \text{ is the observed knowledge graph}
\]

Observation:
\begin{itemize}
  \item Is bounded by the observer's resolution power
  \item Is localized in time and semantic field curvature
  \item Produces distinguishable knowledge points (nodes in a graph)
\end{itemize}

In analogy to quantum measurement, observation is the act of collapsing multiple epistemic possibilities into a single node.

\subsection*{3. Language as Interface}

\textbf{Language} encodes observation into symbols, and thereby transmits structure across agents and contexts. It is the mediating surface between experience and communication.

We define a symbolic encoder:
\[
\mathcal{L} : \mathcal{K} \to \Sigma^*, \quad \text{where } \Sigma^* \text{ is the set of language sequences}
\]

The expressivity and compression of $\mathcal{L}$ are constrained by:
\begin{itemize}
  \item Vocabulary size
  \item Rule density (grammatical transformations)
  \item Semantic anchoring (alignment with underlying $\mathcal{K}$)
\end{itemize}

\subsection*{4. Triadic Flow and Dynamics}

The complete process follows this pipeline:
\[
\mathcal{I} \xrightarrow{\mathcal{O}} \mathcal{K} \xrightarrow{\mathcal{L}} \Sigma^*
\]
Each transformation introduces:
\begin{itemize}
  \item \emph{Irreversibility} (loss due to discretization)
  \item \emph{Energetic or complexity cost} (e.g. minimal action in semantic transitions)
  \item \emph{Potential for drift} (semantic distortion under propagation)
\end{itemize}

\subsection*{5. Applications and Future Structure}

This triadic architecture grounds:
\begin{itemize}
  \item The semantic analog of Planck-scale granularity
  \item Future derivation of constants like $c_\text{sem}$ and $\hbar_\text{lang}$
  \item Directional flow in belief graph evolution
  \item Lossy versus lossless knowledge encoding strategies
\end{itemize}

\end{document}
\input{preamble}
\begin{document}

\section*{Fundamental Constants in Semantic Physics}

This section introduces hypothesized constants that govern the dynamics and structure of semantic systems. Just as physical theories are shaped by constants such as the speed of light $c$ and Planck's constant $\hbar$, semantic theory may be bounded by limits that constrain propagation, granularity, and transformation of knowledge in cognitive and computational systems.

\subsection*{1. $c_{\text{sem}}$: Semantic Propagation Speed}

We define $c_{\text{sem}}$ as the maximum speed at which valid semantic content (truth-preserving knowledge transformations) can propagate through a semantic graph or manifold.

\[
c_{\text{sem}} = \max \left( \frac{d_\text{sem}(v_i, v_j)}{\Delta t} \right)
\]

where $d_\text{sem}$ is the semantic distance between knowledge nodes $v_i$ and $v_j$, and $\Delta t$ is the minimal inference or communication interval.

This is a structural bound, not a computational one. It reflects the fact that semantic resolution and transformation require time and energy, even in idealized systems.

\subsection*{2. $c_{\text{obs}}$: Observation Realization Bound}

Observation is not instantaneous. $c_{\text{obs}}$ is defined as the fastest rate at which observation can instantiate information into a resolved, discrete knowledge state.

\[
c_{\text{obs}} \leq c_{\text{sem}} \leq c
\]

Here $c$ is the physical speed of light, which caps classical signaling. Observation may be bottlenecked by sensory, cognitive, or formal resolution constraints.

\subsection*{3. $\hbar_{\text{lang}}$: Minimum Symbolic Action}

Language does not transmit arbitrary meaning in zero-cost units. We hypothesize the existence of a symbolic quantum, $\hbar_{\text{lang}}$, representing the minimal semantic action needed to produce a distinguishable, transmissible symbolic change.

\[
\delta S \geq \hbar_{\text{lang}} \quad \text{(Minimum action to shift meaning state)}
\]

This places a lower bound on granularity in symbol systems and supports the idea of a minimal “bit” or “token” in semantic systems analogous to Planck length.

\subsection*{4. $G_{\text{sem}}$: Semantic Gravitation Constant}

This constant governs how semantic weight (information density) distorts the local geometry of a belief or knowledge graph.

It is analogous to Newton’s $G$:
\[
F = G_{\text{sem}} \frac{w_i w_j}{d^2}
\]

Here $w_i$ and $w_j$ are the information weights at nodes $i$ and $j$, and $d$ is the graph distance. The force $F$ reflects how much two ideas influence one another based on their semantic mass.

\subsection*{5. Future Formalization}

Each of these constants serves as a hypothesis. They invite:
\begin{itemize}
  \item Experimental estimation (e.g., in LLM drift analysis or belief network dynamics)
  \item Analogy with epistemic phase transitions and state collapse
  \item Formal inclusion in semantic field equations
\end{itemize}

\end{document}
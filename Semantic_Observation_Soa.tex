\documentclass{article}
\usepackage{amsmath}
\usepackage{geometry}
\geometry{margin=1in}

\title{FIL Reference Sheet (Partial Draft)}
\author{}
\date{\emph{Version: Step 1 – Partial, covering Opus\_4 Ch. 1–4 and Cardinality Cascade (3.2.5)}}

\begin{document}

\maketitle

\vspace{1em}
\hrule
\vspace{1em}

\section{Chapter 1 – Foundations}

\subsection{Definition 1.1 – FL Field}
The FL Field is defined as a mapping:
\[
\mathcal{F}_L : \mathcal{O} \times \mathcal{I} \to \mathcal{L}
\]
where:
\begin{itemize}
\item \( \mathcal{O} \): Observer states
\item \( \mathcal{I} \): Information states
\item \( \mathcal{L} \): Laws inferred from the interaction of observer and information
\end{itemize}

\subsection{Theorem 1.1 – Observer–Law Duality}
Given an observer state \(o\) and an information state \(i\), there exists a unique law \(l\) in \( \mathcal{L} \) such that \( \mathcal{F}_L(o,i) = l \). Conversely, for any \(l\) there exists at least one \((o,i)\) pair producing it.

\emph{Proof Sketch:} Follows from surjectivity of \( \mathcal{F}_L \) on \( \mathcal{L} \) and injectivity on \( \mathcal{O} \times \mathcal{I} \) under fixed \(l\).

\hrule

\section{Chapter 2 – Computational Bounds}

\subsection{Definition 2.1 – Computational Light-Speed}
From Landauer and Bremermann bounds:
\[
c_{\mathrm{comp}}(T) = \frac{2 k_B T \ln 2}{\pi \hbar}
\]

\subsection{Theorem 2.1 – Maximum Processing Rate}
Any physical system at temperature \(T\) cannot exceed \(c_{\mathrm{comp}}(T)\) logical operations per second.

\emph{Proof:} Combine Landauer's minimum energy per bit \(E_L = k_B T \ln 2\) with Bremermann's maximum rate \(R_{\max} = 2E/(\pi \hbar)\).

\hrule

\section{Chapter 3 – Semantic Structures}

\subsection{Definition 3.1 – Semantic Temperature}
A scalar field over semantic space indicating the effective processing temperature for a domain.

\subsection{Definition 3.2 – Semantic Distance}
\[
d_{\mathrm{sem}}(L_1, L_2) = \sqrt{[H(L_1|L_2) + H(L_2|L_1)]^2 + (\Delta S_{\mathrm{struct}})^2}
\]

\subsection{Theorem 3.2 – Optimal Bridging Temperature}
\[
T_{\mathrm{opt}}(L_1,L_2) = \frac{\pi \hbar}{2 k_B \ln 2} \, d_{\mathrm{sem}}(L_1, L_2) \, f(\Phi_1, \Phi_2)
\]

\emph{Proof:} Derived from principle of least computational action, balancing activation energy with Boltzmann factor.

\hrule

\section{Section 3.2.5 – Cardinality Cascade}

\subsection{Theorem 3.11 – Computational Generation Bound}
\[
\frac{d|\Lambda(\ell)|}{dt} \leq c_{\mathrm{comp}}(T)
\]

\textbf{Interpretation:} The universe cannot generate distinguishable semantic objects faster than \(c_{\mathrm{comp}}(T)\).

\subsection{Definition 3.10 – Semantic Cardinality Levels}
\begin{itemize}
\item Level 0: \(|P_0| = 2\) (primordial states)
\item Level \(n\): \(|P_n| \leq e^{S_{\max,n} / k_B}\)
\end{itemize}

\subsection{Theorem 3.12 – Entropy–Cardinality Correspondence}
\[
|\Lambda(\ell_n)| \leq \exp \left( \frac{1}{k_B} \int_0^t c_{\mathrm{comp}}(T(\tau)) \, d\tau \right)
\]

\emph{Proof:} Each object generation dissipates \(k_B T \ln 2\) entropy. Integrating available processing over time gives the bound.

\subsection{Theorem 3.16 – Physical Incompleteness}
Any finite computational system with total available operations \( \int_0^t c_{\mathrm{comp}}(T) d\tau \) cannot generate all semantically valid self-referential statements.

\emph{Proof:} Diagonal argument applied under finite operation count constraint.

\hrule

\emph{End of Partial Draft}

\end{document}
```

You can copy the LaTeX code above and paste it into a LaTeX editor like Overleaf (https://www.overleaf.com) or use a local LaTeX compiler like pdflatex to generate the PDF file. This conversion preserves the structure, headings, lists, and mathematical equations from the original Markdown.
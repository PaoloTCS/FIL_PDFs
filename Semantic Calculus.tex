% ==========================================================
%  Semantic Calculus: From Token Sums to Integral Language Transformations
%  ----------------------------------------------------------
%  Draft skeleton prepared for Paolo Pignatelli — July 29 2025
% ==========================================================
\documentclass[11pt]{article}

% ---------- packages ----------
\usepackage{amsmath, amssymb, amsthm}
\usepackage{physics}        % \grad, \div, \laplacian, etc.
\usepackage{bm}             % bold math symbols
\usepackage{tikz}
\usepackage{graphicx}
\usepackage{geometry}
\geometry{margin=1in}
\usepackage{hyperref}

% ---------- metadata ----------
\title{Semantic Calculus:\\From Token Sums to Integral Language Transformations}
\author{Paolo Pignatelli\thanks{Independent Researcher, La Jolla, CA.}
        \and ChatGPT o3\thanks{OpenAI.}}
\date{\today}

% ---------- custom commands ----------
\newcommand{\Token}{\mathcal{T}}
\newcommand{\Manifold}{\mathcal{M}}
\newcommand{\Field}{\Phi}
\newcommand{\Kernel}{K}
\newcommand{\Eps}{\varepsilon}

\begin{document}
\maketitle
\begin{abstract}
\noindent
We introduce a \emph{semantic calculus} that replaces the discrete
“token-sum” algebra of contemporary large-language models with a
continuum formalism based on integrals over bounded harmonic manifolds.
Tokens are re-envisioned as localized basis functions (recursive
ideograms) living on a regularized semantic lattice whose infinitesimal
limit forms a differentiable field~$\Field(x)$.  Transformer layers are
reinterpreted as integral (Fredholm/Volterra-type) operators acting on
$\Field$, enabling rigorous definitions of semantic gradient, divergence,
Laplacian, and Green’s functions.  We lay down the mathematical
foundations, propose a hierarchy-aware integral-attention architecture,
and present a toy implementation that demonstrates smooth semantic
interpolation and energy-efficient inference.
\end{abstract}

% ==========================================================
\section{Introduction}
% ==========================================================
\subsection{Motivation: From Algebra to Calculus of Meaning}
\subsection{Historical Lineage: Chinese Ideograms, Harmonic Analysis, and the Origins of Calculus}
\subsection{Contributions and Paper Roadmap}

% ==========================================================
\section{Preliminaries and Notation}
% ==========================================================
\subsection{Discrete Token Algebra in Contemporary LLMs}
\subsection{Bounded Harmonic Lattices}
\subsection{Recursive Ideogram Tokens}
\subsection{Semantic Manifold $\Manifold$ and Coordinate Charts}

% ==========================================================
\section{From Lattice to Field: The Continuum Limit}
% ==========================================================
\subsection{Infinitesimal Cell Scaling ($\delta \to 0$)}
\subsection{Definition of the Semantic Field $\Field : \Manifold \to \mathbb{R}^n$}
\subsection{Measure-Theoretic Foundations and Normalization}
\subsection{Diagram: Lattice $\rightarrow$ Field (TikZ placeholder)}

% ==========================================================
\section{Integral Operators as Transformer Layers}
% ==========================================================
\subsection{Review of Attention as Weighted Sums}
\subsection{Kernelized Attention: $\displaystyle \psi(x)=\int_{\Manifold}\Kernel(x,y)\,\Field(y)\,dy$}
\subsection{Properties of $\Kernel$: locality, bounded support, harmonicity}
\subsection{Hierarchical Gating via Kernel Modulation}
\subsection{Volterra vs.\ Fredholm Forms and Computational Implications}
\subsection{Diagram: Integral-Operator Layer (TikZ placeholder)}

% ==========================================================
\section{Differential Structure on the Semantic Field}
% ==========================================================
\subsection{Semantic Gradient $\grad\Field$ and Drift}
\subsection{Semantic Divergence $\div\Field$ and Convergence Zones}
\subsection{Semantic Laplacian $\laplacian\Field$ as Curvature of Meaning}
\subsection{Green’s Functions and Semantic Propagation}
\subsection{Path Integrals over Meaning Trajectories}

% ==========================================================
\section{Recursive Ideogram Basis Functions}
% ==========================================================
\subsection{Definition and Construction}
\subsection{Orthogonality and Completeness on $\Manifold$}
\subsection{Connection to Wavelets and Spherical Harmonics}
\subsection{Visualization: Ideogram Decomposition (TikZ placeholder)}

% ==========================================================
\section{Practical Architecture: The Harmonia Transformer}
% ==========================================================
\subsection{Overall Stack: Field Encoder $\rightarrow$ Integral Layers $\rightarrow$ Field Decoder}
\subsection{Hierarchy-Aware Integral Attention (HIA)}
\subsection{Dynamic Expansion and Compression Gates}
\subsection{Energy and Complexity Analysis}

% ==========================================================
\section{Toy Implementation and Experiments}
% ==========================================================
\subsection{1-D Harmonic Semantic Line}
\subsection{2-D Ideogram Grid with Localized Basis Modes}
\subsection{Qualitative Results: Smooth Interpolation\,/\,Semantic Denoising}
\subsection{Quantitative Metrics: Approximation Error, Energy Efficiency}

% ==========================================================
\section{Related Work}
% ==========================================================
\subsection{Graph and Manifold Wavelets}
\subsection{Kernel Attentional Approaches (Performer, Linearized Attention, …)}
\subsection{Neural Field Models and Implicit Representations}
\subsection{Logographic NLP and Chinese-Inspired Architectures}

% ==========================================================
\section{Discussion and Future Directions}
% ==========================================================
\subsection{Symbol–Field Duality and Category-Theoretic View}
\subsection{Multimodal Extensions (Vision, Audio as Fields)}
\subsection{Hardware Implications: Analog Integral Cores}
\subsection{Open Problems: Convergence, Expressivity, Scaling Laws}

% ==========================================================
\section{Conclusion}
% ==========================================================
\noindent
We have outlined a pathway from the discrete algebra of tokens to a
differentiable calculus of meaning.  By grounding language in bounded
harmonic manifolds and casting transformer layers as integral operators,
we obtain a mathematically rich, physically intuitive, and potentially
more efficient paradigm for natural-language understanding and
generation.

% ==========================================================
\bibliographystyle{plain}
\bibliography{semantic_calculus_bib}  % <-- create this .bib file
% ==========================================================
\appendix
\section{Formal Proofs}
\subsection{Completeness of Ideogram Basis}
\subsection{Boundedness of Integral Attention Operator}

\section{Implementation Details}
\subsection{Numerical Quadrature Schemes}
\subsection{PyTorch Prototype Hyperparameters}

\end{document}
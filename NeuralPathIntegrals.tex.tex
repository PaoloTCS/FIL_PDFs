% Neural Path Integrals and the Semantic Action Principle
% ------------------------------------------------------
% arXiv v1 – July 2025
% Paolo Pignatelli <paolo@verbumtechnologies.com>
% ------------------------------------------------------
\documentclass[11pt]{article}

% --- Packages -------------------------------------------------------------
\usepackage[T1]{fontenc}
\usepackage{lmodern}
\usepackage{microtype}
\usepackage{amsmath, amssymb, amsthm, bm}
\usepackage{graphicx}
\usepackage{hyperref}
\hypersetup{colorlinks=true, linkcolor=blue, citecolor=blue, urlcolor=blue}
\usepackage[numbers,sort&compress]{natbib}
\usepackage{geometry}
\geometry{margin=1in}
\usepackage{enumitem}
\setlist{nosep}

% --- Custom commands ------------------------------------------------------
\newcommand{\E}{\mathbb{E}}
\newcommand{\R}{\mathbb{R}}
\newcommand{\calD}{\mathcal{D}}
\newcommand{\calL}{\mathcal{L}}
\newcommand{\calP}{\mathcal{P}}
\newcommand{\calH}{\mathcal{H}}

% --- Meta data ------------------------------------------------------------
\title{Neural Path Integrals and the Semantic Action Principle\[2ex]\
  \large A Physics-Inspired Framework for Energy-Bound Pruning in Deep Networks}
\author{Paolo Pignatelli\\Independent Researcher\\\texttt{paolo@verbumtechnologies.com}}
\date{July 29, 2025}

% --------------------------------------------------------------------------
\begin{document}
\maketitle

\begin{abstract}
We show that the forward inference pass of a gated feed-forward neural network can be written as a discrete Feynman path integral whose action is the (negative) log-product of synaptic weights along an active path.  Promoting this \emph{semantic action} to a physical Hamiltonian density allows us to import energy-based bounds—analogous to the Bekenstein–Bremermann limit—into learning theory.  We derive an \emph{Energy-Constrained Cardinality Cascade} that upper-bounds the number of admissible paths and yields a principled pruning criterion.  A toy ReLU network experiment confirms that the cascade retains predictive accuracy while eliminating \(90\,\%\) of zero-contributing paths.  The framework unifies recent neural-path-kernel results with information-theoretic limits and lays groundwork for hardware designs that implement complex weight phases.\end{abstract}

\tableofcontents

% --------------------------------------------------------------------------
\section{Introduction}
Large language and vision models achieve state-of-the-art performance at the cost of enormous parameter counts and energy budgets.  Yet their computation can be viewed as a superposition of many simpler subnetworks—\emph{paths}.  Inspired by Feynman's sum-over-histories, we ask: can we treat a deep network's inference as a path integral and then impose physical energy constraints to prune superfluous histories?

\section{Background}
\subsection{Feynman path integrals}
[Short recap of the continuous path-integral formalism.]

\subsection{Neural path kernels}
Summarise\cite{chatterjee2020neural,pham2022path} the discrete path perspective on ReLU networks.

\section{The Semantic Action Principle}
Define the effective action \(S[p] = -\hbar \log W[p]\) for a path \(p\) with weight product \(W[p]\).  Show equivalence to a Hamiltonian density \(\mathcal{H}\) over the graph.

\section{Energy-Constrained Cardinality Cascade}
\subsection{Physical incompleteness bound}
Derive \(N_{\text{paths}} \le C E_{\text{tot}} / (k_B T)\).  Connect to Bekenstein bound.
\subsection{Cascade algorithm}
Present Algorithm 1 pseudo-code.

\section{Experiment: ReLU Toy Model}
Describe a 3-layer network, path enumeration, pruning rates, accuracy table.

\section{Discussion}
Links to belief-graph drift, masking fields, and hardware outlook.

\section{Conclusion}
Summarise contributions and future work.

\section*{Acknowledgements}
This manuscript was co-developed through iterative drafting with OpenAI’s GPT-o3 system. All derivations and experiments were verified by the human author.

% --------------------------------------------------------------------------
\bibliographystyle{plainnat}
\bibliography{references}

\end{document}